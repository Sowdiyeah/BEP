\section{Introduction}\label{s:intro}
Applications of automatic digit and character recognition are everywhere. 
Examples are recognising street signs, digitising old documents, or searching for words that appear in a photograph. 
The potential in this field is enormous, and as such there is also a lot of research done in this area~\cite{Anthonissen, Savas2007, Ciresan2010, Cao2006}. 
The automatic classification of handwritten digits is often considered as a standard problem in pattern recognition, because many challenges of pattern recognition are present~\cite{Savas2007}.
The research done in this area led to a number of ways to solve this problem. 
This thesis will give an overview of some of those popular techniques and compare them.

To compare the various techniques, we used labelled images from a subset of the US Postal Service Database which was packaged with a book from Hastie et al~\cite{Hastie2009}. 
This data set is split up into 2 sets. 
There is a training set, which consists of a digit and an image representing that digit. 
This is the only set which the algorithms and models have complete access to. 
The other set is called a test set. 
This set also consists of a digit and an image representing that digit; however in this case the algorithms and models only have access to the images. 
This set is given as input after which a classification attempt is made by using the technique under consideration on the test set. 
Then the corresponding digit in the test set can be used to decide whether or not the classification is correct.

Each image in the data set has a dimension of 16 \(\times \) 16 and thus consists of 256 pixels.
The image is a black-and-white monochrome image, where each pixel has a certain gray-scale value between \(-1\) and \(1\) with a step size of \(5\cdot 10^{-4}\). 
In this paper the lowest pixel value in an image is chosen to be completely white and the highest pixel value black. 
This is merely a convention such that the black digits are displayed on a white background.

The following structure is present in the paper.
First, in Section~\ref{s:knn}, test images will be compared with the training images with the \(k\)-nearest neighbour algorithm.
Secondly, in Section~\ref{s:svd_all}, test images are classified on their distance to subspaces spanned by the training images for a particular number.
Finally, in Section~\ref{s:neural_network}, images are classified through the use of a neural network model.

Section~\ref{s:knn} starts with the remark that each image can be represented by a point in a 256-dimensional space.
In this space each element of the vector corresponds to the brightness of a pixel, hence this space will be called the pixel space.
In this section, the similarity between two images is made more concrete.
Then the idea is to classify a test digit as the digit corresponding to the most similar digit.

Section~\ref{s:svd_all} exploits the fact that all training images for a particular number form a subspace. Using the singular value decomposition this space is represented in a more convenient way.
The idea is to classify a test digit as the digit corresponding to the closest subspace.

In Section~\ref{s:neural_network} we train a model on the training images in the hope that patterns pertaining to different images are extracted.
Test images are then given as input to the model, where the model can use the different patterns to recognise a digit.
Fundamental parts of a neural network will be explained with the use of a small example, after which it will be applied to the digit recognition problem.

Finally, we draw some conclusions in Section~\ref{s:conclusion}.
In the appendix the reader can find parts of the code created in MATLAB 2019a.
